\documentclass{amsart}

\title{Learning Outcomes: 1151}
\begin{document}
\maketitle


\paragraph*{Review of functions}
\begin{itemize}
\item interpret a function from an algebraic, numerical, graphical and
  verbal perspective and extract information relevant to the
  phenomenon modeled by the function.
\item find points of discontinuity for functions and classify them.
\item understand how restricting the domain of a function allows an
  inverse on this restricted domain.
\end{itemize}

\paragraph*{Idea of limits}
\begin{itemize}
\item use a graph to compute limits. 
\item interpret real-word examples of limits, via average velocity and
  instantaneous velocity.
\end{itemize}


\paragraph*{Definition of limits}
\begin{itemize}
\item verify the value of the limit of a function at a point using the
  definition of the limit.
\item interpret a limit using the notion of ``zooming.''
\end{itemize}

\paragraph*{Limit laws}
\begin{itemize}
\item carefully apply limit laws to compute a limit.
\item connect limit laws to the evaluation of limits involving
  polynomials.
\item quantitatively compute limits. 
\end{itemize}

\paragraph*{Infinite limits}
\begin{itemize}
\item interpret functions whose limits are unbounded.
\end{itemize}

\paragraph*{Limits at infinity}
\begin{itemize}
\item compute horizontal and vertical asymptotes. 
\item correctly identify horizontal and vertical asymptotes. 
\end{itemize}


\paragraph*{Continuity}
\begin{itemize}
\item reconcile the difference between a naive understanding of
  continuity and the definition of continutity in terms of limits.
\item understand the consequences of the intermediate value theorem
  for continuous functions.
\end{itemize}

\paragraph*{Introduction to the derivative}
\begin{itemize}
\item interpret the derivative of a function at a point the as the
  instantaneous rate of change in the quantity modeled and state its
  units.
\item interpret the derivative of a function at a point as the slope
  of the tangent line and estimate its value from the graph of a
  function.
\item sketch the graph of the derivative from the given graph of a function.
\item given a table of function values, approximate the value of the
  derivative at a point using the difference quotient.
\item compute the value of the derivative at a point algebraically
  using the (limit) definition.
\item derive the expression for the derivative of elementary functions
  from the (limit) definition.
\item be able to show whether a function is differentiable at a point.
\item compute the expression for the line tangent to a function at a point.
\end{itemize}

\paragraph*{Rules of differentiation}
\begin{itemize}
\item compute the expression for the derivative of a function using
  the rules of differentiation.
\end{itemize}

\paragraph*{Product and quotient rules}
\begin{itemize}
\item compute the expression for the derivative of a function using
  the rules of differentiation.
\item interpret this rule as an extension of the product of two
  binomials.
\end{itemize}

\paragraph*{Derivatives of trig functions}
\begin{itemize}
\item compute the expression for the derivative of a function using
  the rules of differentiation.
\end{itemize}

\paragraph*{Derivative as rates of change}
\begin{itemize}
\item interpret the derivative of a function at a point the as the
  instantaneous rate of change in the quantity modeled and state its
  units.
\item interpret position, velocity, and acceleration in terms of
  calculus.
\end{itemize}

\paragraph*{The chain rule}
\begin{itemize}
\item compute the expression for the derivative of a function using the rules of differentiation.
\item compute the expression for the derivative of a composite
  function using the chain rule of differentiation.
\end{itemize}

\paragraph*{Implicit differentiation}
\begin{itemize}
\item differentiate a relation implicitly and compute the line tangent
  to its graph at a point. 
\end{itemize}

\paragraph*{Derivatives of exponential and logarithmic functions}
\begin{itemize}
\item derive formulas for these derivatives as an application of implicit differentiation. 
\end{itemize}

\paragraph*{Derivatives of inverse trig functions}
\begin{itemize}
\item derive formulas for these derivatives as an application of implicit differentiation. 
\end{itemize}


\paragraph*{Related rates}
\begin{itemize}
\item set up complex word problems.
\item apply techniques from implicit differentiation.
\end{itemize}


\paragraph*{Maxima and minima}
\begin{itemize}
\item compute the critical points of a function on an interval.
\item interpret the value of the first and second derivative as
  measures of increase and concavity of a functions.
\item identify the extrema of a function on an interval and classify them as
 minima, maxima or saddles using the first derivative test.
\end{itemize}


\paragraph*{What derivatives tell us}
\begin{itemize}
\item obtain expressions for higher order derivatives of a function using the rules of
 differentiation.
\item interpret higher order derivatives. 
\end{itemize}


\paragraph*{Graphing functions}
\begin{itemize}
\item compute the critical points of a function on an interval.
\item interpret the value of the first and second derivative as
  measures of increase and concavity of a functions.
\item identify the extrema of a function on an interval and classify them as
 minima, maxima or saddles using the first derivative test.
\end{itemize}


\paragraph*{Optimization problems}
\begin{itemize}
\item apply basic optimization techniques to selected problems arising
  in various fields such as physical modeling, economics and
  population dynamics.
\end{itemize}

\paragraph*{Linear approximation and differentials}
\begin{itemize}
\item interpret the tangent line geometrically as the local linearization of a function.
\item interpret $dx$ and $dy$ formally and intuitively.
\item use the differential to determine the error of approximations.
\end{itemize}

\paragraph*{Mean value theorem}
\begin{itemize}
\item understand the consequences of Rolle's theorem and the Mean
  Value theorem for differentiable functions.
\end{itemize}

\paragraph*{L'Hospitals rule}
\begin{itemize}
\item calculate the limit of a function at a point numerically and algebraically using
 appropriate techniques including l’Hospital’s rule.
\end{itemize}

\paragraph*{Antiderivatives}
\begin{itemize}
\item find the anti-derivative of elementary polynomials, exponential,
  logarithmic and trigonometric functions.
\end{itemize}

\paragraph*{Approximating areas under curves}
\begin{itemize}
\item interpret the definite integral geometrically as the area under a curve.
\item construct a definite integral as the limit of a Riemann sum.
\item approximate a definite integral using left sum, right sum, and
  midpoint.
\end{itemize}


\paragraph*{Definite integrals}
\begin{itemize}
\item interpret the indefinite integral as a definite integral with variable limit(s).
\item exploit symmetry to compute definite integrals of even/odd functions. 
\end{itemize}


\paragraph*{The fundamental theorem of calculus}
\begin{itemize}
\item evaluate a definite integral using an antiderivative.
\item understand why this is a theorem and not the definition.
\end{itemize}

\paragraph*{Working with integrals}
\begin{itemize}
\item understand the consequences of the Mean Value theorem for
  integrals.
\item exploit symmetry to compute definite integrals of even/odd functions. 
\end{itemize}


\paragraph*{Substitution rule}
\begin{itemize}
\item use substitution to find the antiderivative of a composite function.
\end{itemize}


\paragraph*{Velocity and net change}
\begin{itemize}
\item interpret differentiation and anti-differentiation as inverse operations.
\item view the fundamental theorem of calculus as a theorem about
  distance/displacement.
\end{itemize}




\end{document}
