\documentclass{ximera}

%\colorlet{penColor}{blue!50!black} % Color of a curve in a plot
%\colorlet{gridColor}{gray!50} % Color of grid in a plot
%\colorlet{background}{white} % Color of the page
\newcommand{\ddx}{\frac{d}{dx}}
\newcommand{\dydx}{\frac{dy}{dx}}
\newcommand{\dd}[2][]{\frac{d #1}{d #2}}


\outcome{Understand derivatives of trig functions.}
\outcome{Understand special trig limits.}
\outcome{Understand the cyclic nature of the derivatives of sine and cosine.}
\outcome{Compute derivatives of trig functions.}
\outcome{Compute limits using special trig limits and trig identities.}
\outcome{Compute higher order derivatives of sine and cosine.}

\title{Derivatives of trigonometric functions}

\begin{document}

\begin{abstract}
  Now we will work with our old friends, the trigonometric functions. 
\end{abstract}
\maketitle

Here they are, the trigonometric functions and their derivatives:


\begin{theorem}[The Derivatives of Trigonometric Functions] \hfil
\begin{itemize}
\item $\dd{t} \sin(t) = \cos(t)$.
\item $\dd{t} \cos(t) = -\sin(t)$.
\item $\dd{t} \tan(t) = \sec^2(t)$.
\item $\dd{t} \sec(t) = \sec(t)\tan(t)$.
\item $\dd{t} \csc(t) = -\csc(t)\cot(t)$.
\item $\dd{t} \cot(t) = -\csc^2(t)$.
\end{itemize}
\end{theorem}


\begin{question}
  Differentiate $f(t) = \sin(t) + \cos(t)$ with respect to $t$.
  \begin{prompt}
    $\dd{t} f(t) = $\answer{$\cos(t)-\sin(t)$}
  \end{prompt}
\end{question}

\begin{question}
  Differentiate $f(t) = \sin(t)\cdot\cos(t)$ with respect to $t$.
  \begin{prompt}
    $\dd{t} f(t) = $\answer{$\cos^2(t) - \sin^2(t)$}
  \end{prompt}
\end{question}


\begin{question}
  Differentiate $f(t) = \sec(t) + \csc(t)$ with respect to $t$.
  \begin{prompt}
    $\dd{t} f(t) = $\answer{$\sec(t)*\tan(t) - \cot(t)*\csc(t)$}
  \end{prompt}
\end{question}

\begin{question}
  Differentiate $f(t) = \sec(t)\cdot\csc(t)$ with respect to $t$.
  \begin{prompt}
    $\dd{t} f(t) = $\answer{$\sec^2(t) - \csc^2(t)$}
  \end{prompt}
\end{question}

\begin{question}
Write down at least \textbf{five} questions for this lecture. After
you have your questions, label them as ``Level 1,'' ``Level 2,'' or ``Level 3'' where:
\begin{description}
\item[Level 1] Means you know the answer, or know exactly how to do this problem.
\item[Level 2] Means you think you know how to do the problem, or will soon learn how to do the problem.
\item[Level 3] Means you have no idea how to do the problem. 
\end{description}
  \begin{freeResponse}
  \end{freeResponse}
\end{question}

\end{document}
