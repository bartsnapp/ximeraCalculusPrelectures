\documentclass{ximera}

\colorlet{penColor}{blue!50!black} % Color of a curve in a plot
\colorlet{gridColor}{gray!50} % Color of grid in a plot
\colorlet{background}{white} % Color of the page


\outcome{Understand derivatives of trig functions.}
\outcome{Understand special trig limits.}
\outcome{Understand the cyclic nature of the derivatives of sine and cosine.}
\outcome{Compute derivatives of trig functions.}
\outcome{Compute limits using special trig limits and trig identities.}
\outcome{Compute higher order derivatives of sine and cosine.}

\title{derivatives of trig functions}

\begin{document}

\begin{abstract}
  Now we will work with our old friends, the trigonometric functions. 
\end{abstract}
\maketitle

Here they are, the trigonometric functions and their derivatives:


\begin{theorem}[The Derivatives of Trigonometric Functions] \hfil
\begin{itemize}
\item $\dd{\theta} \sin(\theta) = \cos(\theta)$.
\item $\dd{\theta} \cos(\theta) = -\sin(\theta)$.
\item $\dd{\theta} \tan(\theta) = \sec^2(\theta)$.
\item $\dd{\theta} \sec(\theta) = \sec(\theta)\tan(\theta)$.
\item $\dd{\theta} \csc(\theta) = -\csc(\theta)\cot(\theta)$.
\item $\dd{\theta} \cot(\theta) = -\csc^2(\theta)$.
\end{itemize}
\end{theorem}


\begin{question}
  Differentiate $f(\theta) = \sin(\theta) + \cos(\theta)$ with respect to $\theta$.
  \begin{prompt}
    $\dd{\theta} f(\theta) = $\answer{$\cos(\theta)-\sin(\theta)$}
  \end{prompt}
\end{question}

\begin{question}
  Differentiate $f(\theta) = \sin(\theta)\cdot\cos(\theta)$ with respect to $\theta$.
  \begin{prompt}
    $\dd{\theta} f(\theta) = $\answer{$\cos^2(\theta) - \sin^2(\theta)$}
  \end{prompt}
\end{question}


\begin{question}
  Differentiate $f(\theta) = \sec(\theta) + \csc(\theta)$ with respect to $\theta$.
  \begin{prompt}
    $\dd{\theta} f(\theta) = $\answer{$\sec(\theta)*\tan(\theta) - \cot(\theta)*\csc(\theta)$}
  \end{prompt}
\end{question}

\begin{question}
  Differentiate $f(\theta) = \sec(\theta)\cdot\csc(\theta)$ with respect to $\theta$.
  \begin{prompt}
    $\dd{\theta} f(\theta) = $\answer{$\sec^2(\theta) - \csc^2(\theta)$}
  \end{prompt}
\end{question}

\begin{question}
Write down at least \textbf{five} questions for this lecture. After
you have your questions, label them as ``Level 1,'' ``Level 2,'' or ``Level 3'' where:
\begin{description}
\item[Level 1] Means you know the answer, or know exactly how to do this problem.
\item[Level 2] Means you think you know how to do the problem, or will soon learn how to do the problem.
\item[Level 3] Means you have no idea how to do the problem. 
\end{description}
  \begin{freeResponse}
  \end{freeResponse}
\end{question}

\end{document}
