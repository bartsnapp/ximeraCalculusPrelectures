\documentclass{ximera}

\outcome{Limit laws must be used to calculate the exact value of a limit.}
\outcome{Understand what is meant by the form of a limit.}
\outcome{Understand the Squeeze Theorem and how it can be used to find limit values.}
\outcome{Calculate limits using the limit laws.}
\outcome{Calculate limits of the form $0/0$.}
\outcome{Calculate limits of piecewise functions.}
\outcome{Calculate limits using the Squeeze Theorem.}

\title{Limit laws}

\begin{document}

\begin{abstract}
  Limit laws help us compute limits without using the definition.
\end{abstract}

\maketitle

Here is a basic list of limit laws:

\begin{theorem}[Limit Laws]
Suppose that $\lim_{x\to a}f(x)=L$, $\lim_{x\to a}g(x)=M$, $k$
is some constant, and $n$ is a positive integer.
\begin{description}
\item[\textbf{Constant Law}] $\lim_{x\to a} kf(x) = k\lim_{x\to a}f(x)=kL$.
\item[\textbf{Sum Law}] $\lim_{x\to a} (f(x)+g(x)) = \lim_{x\to a}f(x)+\lim_{x\to a}g(x)=L+M$.  
\item[\textbf{Product Law}] $\lim_{x\to a} (f(x)g(x)) = \lim_{x\to a}f(x)\cdot\lim_{x\to a}g(x)=LM$. 
\item[\textbf{Quotient Law}] $\lim_{x\to a} \frac{f(x)}{g(x)} =
  \frac{\lim_{x\to a}f(x)}{\lim_{x\to a}g(x)}=\frac{L}{M}$, if $M\ne0$.
\item[\textbf{Power Law}] $\lim_{x\to a} f(x)^n = \left(\lim_{x\to a}f(x)\right)^n=L^n$.
\item[\textbf{Root Law}] $\lim_{x\to a} \sqrt[n]{f(x)} = \sqrt[n]{\lim_{x\to
    a}f(x)}=\sqrt[n]{L}$ provided if $n$ is even, then $f(x)\ge 0$
  near $a$.
\item[\textbf{Composition Law}] If $\lim_{x\to a}g(x)=M$ and
  $\lim_{x\to M}f(x) = f(M)$, then $\lim_{x\to a} f(g(x)) = f(M)$.
\end{description}
\end{theorem}


Limit laws allow us to ``cut out'' irrelevant data when computing
limits and focus on the issues at hand.

A students you should be \textbf{aware} that these limit laws exist,
but typically you do not have to state them each time you use them,
\textbf{unless} we ask you to.

\begin{question}
  Suppose I tell you that $\lim_{x\to 3} f(x) = -4$. Can you use the
  limit laws above to evaluate the following limit?

\[
\lim_{x\to 3} \left(7f(x)^4 - 14f(x)^2 + 17\right)
\]

    \begin{multipleChoice}
      \choice[correct]{Yes you can.}
      \choice{No you cannot.}
    \end{multipleChoice}  
\end{question}




\begin{question}
  Suppose I tell you that $\lim_{x\to -6} f(x) = 1$. Can you use the
  limit laws above to evaluate the following limit?

\[
\lim_{x\to -6} -13\sqrt{f(x)-1}
\]

    \begin{multipleChoice}
      \choice[correct]{Yes you can.}
      \choice{No you cannot.}
    \end{multipleChoice}  
\end{question}




\begin{question}
  Suppose I tell you that $\lim_{x\to -1} f(x) = 0$ and $\lim_{x\to
    -1}g(x) = 0$. Can you use the limit laws above to evaluate the
  following limit?

\[
\lim_{x\to -1} \frac{14 f(x)^2}{x\cdot g(x)}
\]

    \begin{multipleChoice}
      \choice[correct]{No you cannot.}
      \choice{Yes you can.}
    \end{multipleChoice}  
\end{question}



\begin{question}
  Suppose I tell you that $\lim_{x\to 5} f(x) = 17$ and $\lim_{x\to
    5}g(x) = -22$. Can you use the limit laws above to evaluate the
  following limit?

\[
\lim_{x\to 5} \frac{f(x)-17}{g(x)+22}
\]

    \begin{multipleChoice}
      \choice[correct]{No you cannot.}
      \choice{Yes you can.}
    \end{multipleChoice}  
\end{question}




\begin{question}
Write down at least \textbf{five} questions for this lecture. After
you have your questions, label them as ``Level 1,'' ``Level 2,'' or ``Level 3'' where:
\begin{description}
\item[Level 1] Means you know the answer, or know exactly how to do this problem.
\item[Level 2] Means you think you know how to do the problem, or will soon learn how to do the problem.
\item[Level 3] Means you have no idea how to do the problem. 
\end{description}
\begin{freeResponse}
\end{freeResponse}
\end{question}

\end{document}
