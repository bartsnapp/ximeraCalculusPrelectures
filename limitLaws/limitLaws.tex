\documentclass{ximera}

\outcome{Limit laws must be used to calculate the exact value of a limit.}
\outcome{Understand what is meant by the form of a limit.}
\outcome{Understand the Squeeze Theorem and how it can be used to find limit values.}
\outcome{Calculate limits using the limit laws.}
\outcome{Calculate limits of the form $0/0$.}
\outcome{Calculate limits of piecewise functions.}
\outcome{Calculate limits using the Squeeze Theorem.}

\title{Limit laws}

\begin{document}

\begin{abstract}
  Limit laws help us compute limits without using the definition.
\end{abstract}

\maketitle

Here is a basic list of limit laws:

\begin{theorem}[Limit Laws]
Suppose that $\lim_{x\to a}f(x)=L$, $\lim_{x\to a}g(x)=M$, $k$
is some constant, and $n$ is a positive integer.
\begin{description}
\item[\textbf{Constant Law}] $\lim_{x\to a} kf(x) = k\lim_{x\to a}f(x)=kL$.
\item[\textbf{Sum Law}] $\lim_{x\to a} (f(x)+g(x)) = \lim_{x\to a}f(x)+\lim_{x\to a}g(x)=L+M$.  
\item[\textbf{Product Law}] $\lim_{x\to a} (f(x)g(x)) = \lim_{x\to a}f(x)\cdot\lim_{x\to a}g(x)=LM$. 
\item[\textbf{Quotient Law}] $\lim_{x\to a} \frac{f(x)}{g(x)} =
  \frac{\lim_{x\to a}f(x)}{\lim_{x\to a}g(x)}=\frac{L}{M}$, if $M\ne0$.
\item[\textbf{Power Law}] $\lim_{x\to a} f(x)^n = \left(\lim_{x\to a}f(x)\right)^n=L^n$.
\item[\textbf{Root Law}] $\lim_{x\to a} \sqrt[n]{f(x)} = \sqrt[n]{\lim_{x\to
    a}f(x)}=\sqrt[n]{L}$ provided if $n$ is even, then $f(x)\ge 0$
  near $a$.
\item[\textbf{Composition Law}] If $\lim_{x\to a}g(x)=M$ and
  $\lim_{x\to M}f(x) = f(M)$, then $\lim_{x\to a} f(g(x)) = f(M)$.
\end{description}
\end{theorem}





\begin{question}
  What is the correct answer to this question?
    \begin{multipleChoice}
      \choice[correct]{Correct answer}
      \choice{First Distractor}
      \choice{Second Distractor}
      \choice{Third Distractor}
    \end{multipleChoice}  
\end{question}


\begin{question}
What other questions do you have about this lecture?
\begin{freeResponse}
\end{freeResponse}
\end{question}

\end{document}
