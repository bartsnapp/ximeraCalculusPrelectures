\documentclass{ximera}

%\colorlet{penColor}{blue!50!black} % Color of a curve in a plot
%\colorlet{gridColor}{gray!50} % Color of grid in a plot
%\colorlet{background}{white} % Color of the page
\newcommand{\ddx}{\frac{d}{dx}}
\newcommand{\dydx}{\frac{dy}{dx}}
\newcommand{\dd}[2][]{\frac{d #1}{d #2}}


\outcome{Recall inverse trigonometric functions; their meaning, domain, etc.}
\outcome{Derive the derivatives of inverse trig functions.}
\outcome{How does the derivative of an inverse function relate to the original derivative?}
\outcome{Take derivatives which involve inverse trig functions.}
\outcome{Find derivatives of inverse functions in general.}

\title{Derivatives of inverse trigonometric functions}

\begin{document}

\begin{abstract}
 We can find the derivatives of inverse functions using implicit differentiation.
\end{abstract}

\maketitle

How can we compute the derivative of the inverse of a function? The idea is to use implicit differentiation. 

\begin{align*}
	f(f^{-1}(x)) &= x  && \text{by the definition of inverse functions} \\
	\frac{d}{dx} f(f^{-1}(x)) &=  \frac{d}{dx} x &&\text{implicit differentiation}\\
	 f'(f^{-1}(x))\frac{d}{dx}[ f^{-1}(x)] &= 1 &&\text{chain rule}\\
	 \frac{d}{dx}[ f^{-1}(x)] &= \frac{1}{f'(f^{-1}(x))}
\end{align*}

The result could also be written 

\[
\frac{d}{dx}[ f^{-1}(x)] = \frac{1}{\frac{d}{dt}\left[f(t)\right]\big|_{t=f^{-1}(x)}}
\]

\begin{question}
	Given the following table of values for $f$ and its derivative, compute $\frac{d}{dx}\left[ f^{-1}(x)\right]\big|_{x=3}$
	
	\[
\begin{array}{|c||c|c|c|c|c|}
\hline
 x    & 1 & 2 & 3 & 4 & 5 \\ \hline \hline 
f(x)  & 2 & 3 & 7 & 8 & 12 \\ \hline
f'(x) & 2 & 4 & 1 & 3 & 4 \\ \hline
\end{array}
\]
	\begin{hint}
		The hardest part is probably to realize that since $f(2)=3$, $f^{-1}(3)=2$.
	\end{hint}
	$\frac{d}{dx}\left[ f^{-1}(x)\right]\big|_{x=3} = $ \answer{1/4}


\end{question}

This formula lets us compute the derivatives of the inverse trig functions.  For example,

\[
	\frac{d}{dx} \arcsin(x) = \frac{1}{\cos(\arcsin(x))}\\
\]

We can use geometry to simplify this expression further.  By the pythagorean theorem, we have

\[
\left[\cos(\arcsin(x))\right]^2+\left[\sin(\arcsin(x))\right]^2=1
\] 

but $\sin(\arcsin(x)) = x$, so we have

\[
\left[\cos(\arcsin(x))\right]^2+x^2=1
\]

Solving this for $\cos(\arcsin(x))$ we obtain $\cos(\arcsin(x))=\sqrt{1-x^2}$.

So we can conclude

\[
\frac{d}{dx} \arcsin(x) = \frac{1}{\sqrt{1-x^2}}
\]

A similar argument yields all of the following identities.

\begin{itemize}
\item $\frac{d}{dx} \arcsin(x) = \frac{1}{\sqrt{1-x^2}}$
\item $\frac{d}{dx} \arccos(x) = \frac{-1}{\sqrt{1-x^2}}$
\item $\frac{d}{dx} \arctan(x) = \frac{1}{\sqrt{1+x^2}}$
\item $\frac{d}{dx} \arccot(x) = \frac{-1}{\sqrt{1+x^2}}$
\item $\frac{d}{dx} \arcsec(x) = \frac{1}{|x|\sqrt{x^2-1}}$
\item $\frac{d}{dx} \arccsc(x) = \frac{-1}{|x|\sqrt{x^2-1}}$
\end{itemize}

\begin{question}
	Compute $\frac{d}{dx} \arcsin(x^2)$
		\begin{hint}
			Use the chain rule.
		\end{hint}
		$\frac{d}{dx} \arcsin(x^2)=$\answer{2x/(sqrt(1-x^4))}
\end{question}

\begin{question}
Write down at least \textbf{five} questions for this lecture. After
you have your questions, label them as ``Level 1,'' ``Level 2,'' or ``Level 3'' where:
\begin{description}
\item[Level 1] Means you know the answer, or know exactly how to do this problem.
\item[Level 2] Means you think you know how to do the problem, or will soon learn how to do the problem.
\item[Level 3] Means you have no idea how to do the problem. 
\end{description}
  \begin{freeResponse}
  \end{freeResponse}
\end{question}

\end{document}
