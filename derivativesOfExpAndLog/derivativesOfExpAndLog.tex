\documentclass{ximera}

\outcome{What is the derivative of a log function?}
\outcome{Can you use the power rule when there is a variable in the exponent?  Why or why not?}
\outcome{How can logs be used to help find derivatives?}
\outcome{Take derivatives of logarithms and exponents of all bases.}
\outcome{Use logarithmic differentiation to simplify taking derivatives.}
\outcome{Take derivatives of functions raised to functions.}
\outcome{Apply the generalized power rule.}

\title{Derivatives of exp and log}

\begin{document}

\begin{abstract}
  A few algebraic tricks will let us take derivatives of logarithms and complicated exponential functions.
\end{abstract}

\maketitle

To find the derivative of the natural log function, we use the fact
that it is the inverse of the natural exponential function (which we
know how to differentiate), and implicit differentiation.

\begin{align*}
y&=\ln(x)\\
e^y &= e^{\ln(x)}\\
e^y &= x\\                      
\frac{d}{dx} e^y &= \frac{d}{dx} x\\
e^y\frac{dy}{dx} &= 1\\
x\frac{dy}{dx} &= 1 \text{ since $e^y=x$ by the third line }\\
\frac{dy}{dx} &= \frac{1}{x}
\end{align*}

Thus we have found that $\frac{d}{dx} \ln(x) = \frac{1}{x}$

\begin{question}
	Compute $\frac{d}{dx} \ln(1+x^2)$
	\begin{hint}
	  Use the chain rule.
	\end{hint}
	$\frac{d}{dx} \ln(1+x^2)=$\answer{2x/(1+x^2)}
\end{question}

The most important computation in this section is the following computation, which is not  calculus, but pure algebra:

\[
u^v = e^(\ln(u^v)) = e^(v\ln(u))
\]

All of the results of this section follow from this one unifying formula, so there is no need to memorize a whole bunch of formulas.   We will refer to it as "the unifying formula" for the rest of this prelecture.


\begin{question}
  Compute $\frac{d}{dx} 3^x $
  \begin{hint}
    Rewrite $3^x$ as $e^{x\ln(3)}$ and use the chain rule.
  \end{hint}
  $\frac{d}{dx} 3^x =$ \answer{log(3)*3^x} 
\end{question}

\begin{question}
  Compute $\frac{d}{dx} x^x$.
  
  \begin{hint}
    Rewrite $x^x$ as $e^{x\ln(x)}$.
  \end{hint}
  \begin{hint}
    Then use the chain rule, followed by the product rule.
  \end{hint}
  $\frac{d}{dx} (1+\frac{x^2}{100})^x = $ \answer{x^x*(1+log(x))}
\end{question}

The unifying formula lets us express any exponential function in terms of the natural exponential function.  It also lets us write any logarithm in terms of the natural logarithm:

\begin{align*}
	y &= \ln_b(x)\\
	b^y &=  x\\
	e^{y\ln(b)} &= e^{\ln(x)} \text{ where we use the "unifying formula"}\\
	y\ln(b) &= \ln(x)\\
	y &= \frac{\ln(x)}{\ln(b)}
\end{align*}

So we can conclude that $\log_b(x) = \frac{\ln(x)}{\ln(b)}$.

\begin{question}
	Compute $\frac{d}{dx} log_2(x)$.
	\begin{hint}
	  Rewrite $\log_2(x)$ as $\frac{\ln(x)}{\ln(2)}$ and differentiate.
	\end{hint}
	\begin{hint}
	  $\ln(2)$ is just a constant!
	\end{hint}
	$\frac{d}{dx} log_2(x)=$\answer{1/(x*log(2))}
\end{question}


\begin{question}

Write down at least \textbf{five} questions for this lecture. After
you have your questions, label them as ``Level 1,'' ``Level 2,'' or ``Level 3'' where:
\begin{description}
\item[Level 1] Means you know the answer, or know exactly how to do this problem.
\item[Level 2] Means you think you know how to do the problem, or will soon learn how to do the problem.
\item[Level 3] Means you have no idea how to do the problem. 
\end{description}
  \begin{freeResponse}
  \end{freeResponse}
\end{question}

\end{document}
