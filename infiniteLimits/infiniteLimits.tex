\documentclass{ximera}

\outcome{What does it mean for a limit to equal infinity?}
\outcome{What is a vertical asymptote?}
\outcome{What is the relationship between limits and vertical asymptotes?}


\title{Infinite limits}

\begin{document}

\begin{abstract}
  Sometimes when limits don't equal a real number, we can still say a
  bit more.
\end{abstract}
\maketitle

Consider the function

\[
f(x) = \frac{1}{(x+1)^2}
\]
\begin{image}
\colorlet{penColor}{blue!50!black}
\begin{tikzpicture}
	\begin{axis}[
            domain=-2:1,
            ymax=100,
            width=6in,
            samples=100,
            axis lines =middle, xlabel=$x$, ylabel=$y$,
            every axis y label/.style={at=(current axis.above origin),anchor=south},
            every axis x label/.style={at=(current axis.right of origin),anchor=west}
          ]
	  \addplot [very thick, penColor, smooth, domain=(-2:-1.1)] {1/(x+1)^2};
          \addplot [very thick, penColor, smooth, domain=(-.9:1)] {1/(x+1)^2};
          \addplot [black, dashed] plot coordinates {(-1,0) (-1,100)};
        \end{axis}
\end{tikzpicture}
\end{image}
While the $\lim_{x\to -1} f(x)$ does not exist, something can still be
said.

\begin{definition}
If $f(x)$ grows arbitrarily large as $x$ approaches $a$, we write
\[
\lim_{x\to a} f(x) = \infty
\]
and say that the limit of $f(x)$ \textbf{approaches infinity} as $x$
goes to $a$.

If $|f(x)|$ grows arbitrarily large as $x$ approaches $a$ and $f(x)$ is
negative, we write
\[
\lim_{x\to a} f(x) = -\infty
\]
and say that the limit of $f(x)$ \textbf{approaches negative infinity}
as $x$ goes to $a$.
\end{definition}


\begin{question}
  Evaluate

  \[
  \lim_{x\to 3} \frac{x+3}{(x-3)^2}.
  \]
  
  \begin{multipleChoice}
    \choice[correct]{This limit approaches infinity.}
    \choice{This limit approaches negative infinity.}
    \choice{This limit does not exist.}
    \choice{This limit equals $\frac{1}{6}$.}
  \end{multipleChoice}
\end{question}


\begin{question}
  Evaluate

  \[
  \lim_{x\to 3} \frac{x+3}{(x-3)^3}.
  \]
  
  \begin{multipleChoice}
    \choice[correct]{This limit does not exist.}
    \choice{This limit approaches negative infinity.}
    \choice{This limit approaches infinity.}
    \choice{This limit equals $\frac{1}{9}$.}
  \end{multipleChoice}  
\end{question}



\begin{question}
  Evaluate

  \[
  \lim_{\theta\to \pi/2} |\sec(\theta)|.
  \]
  
  \begin{multipleChoice}
    \choice[correct]{This limit does not exist.}
    \choice{This limit approaches negative infinity.}
    \choice{This limit approaches infinity.}
    \choice{This limit equals $\frac{1}{9}$.}
  \end{multipleChoice}  
\end{question}


\begin{question}
  Evaluate

  \[
  \lim_{\theta\to \pi/2} |\sec(\theta)|.
  \]
  
  \begin{multipleChoice}
    \choice[correct]{This limit approaches infinity.}
    \choice{This limit approaches negative infinity.}
    \choice{This limit does not exist.}
    \choice{This limit equals $1$.}
  \end{multipleChoice}  
\end{question}


\begin{question}
  Evaluate

  \[
  \lim_{\theta\to \pi/2} \sec(\theta).
  \]
  
  \begin{multipleChoice}
    \choice[correct]{This limit does not exist.}
    \choice{This limit approaches negative infinity.}
    \choice{This limit approaches infinity.}
    \choice{This limit equals $1$.}
  \end{multipleChoice}  
\end{question}



\begin{question}
Write down at least \textbf{five} questions for this lecture. After
you have your questions, label them as ``Level 1,'' ``Level 2,'' or
``Level 3'' where:
\begin{description}
\item[Level 1] Means you know the answer, or know exactly how to do
  this problem.
\item[Level 2] Means you think you know how to do the problem, or will
  soon learn how to do the problem.
\item[Level 3] Means you have no idea how to do the problem.
\end{description}
  \begin{freeResponse}
  \end{freeResponse}
\end{question}

\end{document}
