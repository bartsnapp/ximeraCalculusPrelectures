\documentclass{ximera}

%\colorlet{penColor}{blue!50!black} % Color of a curve in a plot
%\colorlet{gridColor}{gray!50} % Color of grid in a plot
%\colorlet{background}{white} % Color of the page
\newcommand{\ddx}{\frac{d}{dx}}
\newcommand{\dydx}{\frac{dy}{dx}}
\newcommand{\dd}[2][]{\frac{d #1}{d #2}}


\outcome{What is an accumulation function?}
\outcome{How does the derivative of the accumulation function compare to the original function?}
\outcome{How is the area under the curve related to the antiderivative?}
\outcome{How are indefinite and definite integrals related?}
\outcome{Calculate and evaluate accumulation functions.}
\outcome{Take derivatives of accumulation functions using the 1st Fundamental Theorem of Calculus.}
\outcome{Evaluate definite integrals using the 2nd Fundamental Theorem of Calculus.}

\title{fundamental theorem of calculus}

\begin{document}

\begin{abstract}
  The fundamental theorems of calculus connect differential and integral calculus.
\end{abstract}

\maketitle

\begin{theorem}[First Fundamental Theorem of Calculus]
If $f$ is continuous on the interval $[a,b]$, then the function $A(x) = \displaystyle \int_a^x f(t)dt$ satisfies $A'(x) = f(x)$.

In other words, $f(x) = \frac{d}{dx} \displaystyle \int_a^x f(t) \d t$.
\end{theorem}

Rephrasing this in English, we could say "The integral valued function $A$ is an antiderivative of $f$"


\begin{question}
	Let $F(x) = \int_1^x t^2 \d t$.  Then $F'(2)=$ \answer{4}
\end{question}

\begin{question}
	\begin{hint}
		If we let $A(x) = \int_1^x f(t) \d t$, then we are trying to approximate $A(2.1)$.  We know $A(2)=3$, and by the fundamental theorem, we know that $A'(2) = 4$.  So we can use the linear approximation to $A$ at $x=2$ to approximate $A(2.1)$.
	\end{hint}
	If $\int_1^2 f(t) \d t = 3$ and $f(2) = 4$, then $\int_1^{2.1} f(t) \d t $ is approximately \answer{3.4}.   
\end{question}

\begin{question}
\begin{hint}
	Use geometry!
\end{hint}
Let's check the fundamental theorem for a particular example function.

Let $f(t) = t$.  Then $A(x) = \int_0^x f(t) \d t=$\answer{x^2/2}
\end{question}

\begin{question}
	\begin{solution}
	 So differentiating this expression $A'(x) =$\answer{x}
	 \end{solution}
Awesome!  We confirmed the fundamental theorem of calculus, since $A'(x) = f(x)$ in this case.
 
\end{question}



\begin{theorem}[Second Fundamental Theorem of Calculus]
	If $f$ is continuous on $[a,b]$, and $F$ is any antiderivative of $f$, then $\int_a^b f(t) \d t = F(b) - F(a)$.
\end{theorem}

This theorem is a \textbf{really big deal}.  It means that instead of laboriously calculating definite integrals using Riemann Sums, we can evaluate them quickly and efficiently as long as we can find an antiderivative of the integrand.

\begin{question}
	\begin{hint}
		What is an antiderivative of $x^2$?  Evaluate this antiderivative at the two end points, and report the difference.
	\end{hint}
	 $\int_1^2 x^2 \d x =$\answer{7/3}
\end{question}

\begin{question}
	$\int_4^5 \cos(x)=$\answer{sin(5) - sin(4)}
\end{question}



Write down at least \textbf{five} questions for this lecture. After
you have your questions, label them as ``Level 1,'' ``Level 2,'' or ``Level 3'' where:
\begin{description}
\item[Level 1] Means you know the answer, or know exactly how to do this problem.
\item[Level 2] Means you think you know how to do the problem, or will soon learn how to do the problem.
\item[Level 3] Means you have no idea how to do the problem. 
\end{description}
\begin{question}
  \begin{freeResponse}
  \end{freeResponse}
\end{question}

\end{document}
