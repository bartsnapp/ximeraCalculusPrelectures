\documentclass{ximera}

%\colorlet{penColor}{blue!50!black} % Color of a curve in a plot
%\colorlet{gridColor}{gray!50} % Color of grid in a plot
%\colorlet{background}{white} % Color of the page
\newcommand{\ddx}{\frac{d}{dx}}
\newcommand{\dydx}{\frac{dy}{dx}}
\newcommand{\dd}[2][]{\frac{d #1}{d #2}}


\outcome{Recognize a composition of functions.}
\outcome{How do you take the derivative of a composition of functions?}
\outcome{How does rate of change work when quantities are dependent upon each other?}
\outcome{How does order of operations determine how you take derivatives which require multiple derivative rules?}
\outcome{Take derivatives of compositions of functions using the Chain Rule.}
\outcome{Take derivatives that require the use of multiple derivative rules.}

\title{Chain rule}

\begin{document}

\begin{abstract}
  The chain rule tells us how to take the derivative of a composition
  of functions.
\end{abstract}
\maketitle


So far we have seen how to compute the derivative of a function built
up from other functions by addition, subtraction, multiplication and
division. There is another very important way that we combine
functions: composition. The \textit{chain rule} allows us to deal with
this case. Consider
\[
h(x) = (1+2x)^5.
\] 

While there are several different ways to differentiate this function,
if we let $f(x) = x^5$ and $g(x) = 1+2x$, then we can express $h(x) =
f(g(x))$. The question is, can we compute the derivative of a
composition of functions using the derivatives of the constituents
$f(x)$ and $g(x)$? To do so, we need the \textit{chain rule}.

\begin{theorem}[Chain Rule]
If $f(x)$ and $g(x)$ are differentiable, then
\[
\ddx f(g(x)) = f'(g(x))g'(x).
\]
\end{theorem}

\begin{question}
  Given $h(x) = (1+2x)^5$, compute
\[
\ddx h(x).
\]
\begin{prompt}
$\ddx h(x) = $\answer{10*(1+2*x)^4}
\end{prompt}
\end{question}

You can also use the chain rule when you have a table of function
values.
\[
\begin{array}{|c||c|c|c|c|c|}
\hline
 x    & 1 & 2 & 3 & 4 & 5 \\ \hline \hline 
f(x)  & 3 & 4 & 1 & 3 & 1 \\ \hline
f'(x) & 2 & 1 & 4 & 2 & 3 \\ \hline
g(x)  & 5 & 3 & 5 & 1 & 4 \\ \hline
g'(x) & 4 & 5 & 2 & 5 & 2 \\ \hline
\end{array}
\]

\begin{question}
Use the table above to compute
\[
\left.\ddx f(g(x))\right|_{x=2}
\]
\begin{prompt}
$\left.\ddx f(g(x))\right|_{x=2}=$\answer{20}
\end{prompt}
\end{question}


\begin{question}
Use the table above to compute
\[
\left.\ddx g(f(x))\right|_{x=3}
\]
\begin{prompt}
$\left.\ddx g(f(x))\right|_{x=3}=$\answer{16}
\end{prompt}
\end{question}


\begin{question}
Use the table above to compute
\[
\left.\ddx f(f(x))\right|_{x=1}
\]
\begin{prompt}
$\left.\ddx f(f(x))\right|_{x=1}=$\answer{8}
\end{prompt}
\end{question}




\begin{question}
Write down at least \textbf{five} questions for this lecture. After
you have your questions, label them as ``Level 1,'' ``Level 2,'' or ``Level 3'' where:
\begin{description}
\item[Level 1] Means you know the answer, or know exactly how to do this problem.
\item[Level 2] Means you think you know how to do the problem, or will soon learn how to do the problem.
\item[Level 3] Means you have no idea how to do the problem. 
\end{description}
  \begin{freeResponse}
  \end{freeResponse}
\end{question}

\end{document}
