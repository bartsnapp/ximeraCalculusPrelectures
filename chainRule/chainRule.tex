\documentclass{ximera}

\outcome{Recognize a composition of functions.}
\outcome{How do you take the derivative of a composition of functions?}
\outcome{How does rate of change work when quantities are dependent upon each other?}
\outcome{How does order of operations determine how you take derivatives which require multiple derivative rules?}
\outcome{Take derivatives of compositions of functions using the Chain Rule.}
\outcome{Take derivatives that require the use of multiple derivative rules.}

\title{Chain rule}

\begin{document}

\begin{abstract}
  The chain rule tells us how to take the derivative of a composition
  of functions.
\end{abstract}
\maketitle


So far we have seen how to compute the derivative of a function built
up from other functions by addition, subtraction, multiplication and
division. There is another very important way that we combine
functions: composition. The \textit{chain rule} allows us to deal with
this case.

\section{The Chain Rule}


Consider
\[
h(x) = (1+2x)^5.
\] 

While there are several different ways to differentiate this function,
if we let $f(x) = x^5$ and $g(x) = 1+2x$, then we can express $h(x) =
f(g(x))$. The question is, can we compute the derivative of a
composition of functions using the derivatives of the constituents
$f(x)$ and $g(x)$? To do so, we need the \textit{chain rule}.

\begin{theorem}[Chain Rule]
If $f(x)$ and $g(x)$ are differentiable, then
\[
\ddx f(g(x)) = f'(g(x))g'(x).
\]
\end{theorem}

\begin{question}
  What is $\frac{d}{dx} \cos \sin x$?

  
    \begin{multipleChoice}
      \choice[correct]{Correct answer}
      \choice{First Distractor}
      \choice{Second Distractor}
      \choice{Third Distractor}
    \end{multipleChoice}  
\end{question}


\begin{question}
Write down at least \textbf{five} questions for this lecture. After
you have your questions, label them as ``Level 1,'' ``Level 2,'' or ``Level 3'' where:
\begin{description}
\item[Level 1] Means you know the answer, or know exactly how to do this problem.
\item[Level 2] Means you think you know how to do the problem, or will soon learn how to do the problem.
\item[Level 3] Means you have no idea how to do the problem. 
\end{description}
  \begin{freeResponse}
  \end{freeResponse}
\end{question}

\end{document}
