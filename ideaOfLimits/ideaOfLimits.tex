\documentclass{ximera}

\outcome{Interpret real-word examples of limits via average velocity
  and instantaneous velocity}

\outcome{Interpert the slope of secant lines verses the slope of
  tangent lines.}

\outcome{Understand the difference between evaluating functions and
  taking limits of functions.}

\outcome{Using a graph to compute limits.}

\title{The idea of limits}

\begin{document}

\begin{abstract}
  This activity will motivate the need for limits in mathematics.
\end{abstract}
\maketitle

Sometimes we have functions that are not defined at certain
points. For example, consider
\[
f(x) = \frac{x^2 - 3x + 2}{x-2}.
\]
You may be tempted to divide by $x^2 - 3x + 2$ by $x-2$, and conclude
that our function is equal to $x-1$. However this is not true!
\[
f(x) = x-1 \qquad\text{only when $x\ne 2$,}
\]
the function $f(x)$ is \textbf{undefined} when $x= 2$, since if we
attempt to compute $f(2)$ we find
\[
f(2) = \frac{2^2-3\cdot 2+2}{2-2} = \frac{0}{0}\qquad\text{which is  undefined}.
\]
There are other types of functions that are undefined at given points. 

\begin{question}
  Below we have a collection of functions and points. Which
  function(s) is/are defined at the given point?
\begin{solution}
\begin{multiple-choice}
\choice[correct]{$f(x) = \sqrt{x}$ at $x= 0$}
\choice{$f(x) = \frac{1}{x}$ at $x= 0$}
\choice{$f(x) = \ln(x)$ at $x=0$}
\choice{$f(x) = \sin^{-1}(x)$ at $x=2$}
\end{multiple-choice}
\end{solution}
\end{question}


\begin{question}
A graph question
\end{question}




Limits also appear in real world settings. You've known how to compute
average velocity for some time, it is just
\[
\text{average velocity} = \frac{\text{distance}}{\text{time}}.
\]
However, we are typically more interested in velocity at a specific
instant. This is called \textit{instantaneous velocity}.

\begin{question}
  Give a plot and ask about average velocities of smaller and smaller
  settings. Maybe driving from Columbus OH to Myrtle Beach SC. 

  \begin{solution}
    \begin{multiple-choice}
      \choice[correct]{Correct answer}
      \choice{First Distractor}
      \choice{Second Distractor}
      \choice{Third Distractor}
    \end{multiple-choice}  
  \end{solution}
\end{question}

What other questions do you have about this lecture?
\begin{free-response}
Answers will vary.
\end{free-response}

\end{document}
