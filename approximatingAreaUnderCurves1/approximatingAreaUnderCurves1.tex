\documentclass{ximera}

\outcomes{What is area?}
\outcomes{How do we find area under a curve?}
\outcomes{What do all the components of the formula mean?}
\outcomes{What is sigma notation and what does it mean?}
\outcomes{What are grid points and how do they help us compute area?} 
\outcomes{How do we know which rectangle we are on?}
\outcomes{What are the lengths and widths of the rectangles?}
\outcomes{Add up a large number of terms quickly using sigma notation.}
\outcomes{Approximate area under a curve.}
\outcomes{Approximate displacement from velocity.}
\outcomes{Compute left, right, and midpoint Riemann Sums.}

\title{approximating area under curves1}

\begin{document}

\begin{abstract}
  The abstract should be a one sentence summary that states the main point of the activity.
\end{abstract}

\maketitle

\begin{question}
  What is the correct answer to this question?

  
    \begin{multipleChoice}
      \choice[correct]{Correct answer}
      \choice{First Distractor}
      \choice{Second Distractor}
      \choice{Third Distractor}
    \end{multipleChoice}  
  
\end{question}

Write down at least \textbf{five} questions for this lecture. After
you have your questions, label them as ``Level 1,'' ``Level 2,'' or ``Level 3'' where:
\begin{description}
\item[Level 1] Means you know the answer, or know exactly how to do this problem.
\item[Level 2] Means you think you know how to do the problem, or will soon learn how to do the problem.
\item[Level 3] Means you have no idea how to do the problem. 
\end{description}
\begin{question}
  \begin{freeResponse}
  \end{freeResponse}
\end{question}

\end{document}
