\documentclass{ximera}

\outcome{What does it mean to take the limit as $x$ approaches $\infty$?}
\outcome{Determinate and indeterminate forms.}
\outcome{Why is infinity not a number?}
\outcome{Calculate the limit as $x$ approaches $\pm\infty$ of common functions algebraically.}
\outcome{Decide whether a form is determinate or indeterminate.}
\outcome{Find the limit as $x$ approaches $\pm\infty$ from a graph.}

\title{Limits at infinity}
\begin{document}

\begin{abstract}
  This activity will introduce students to the idea of a limit at infinity.
\end{abstract}
\maketitle

So far, when we study limits, we typically have something like this:
\[
\lim_{x\to a} f(x) = L
\]
This means that we can get the value of $f(x)$ as close as we want to
$L$, by moving $x$ closer and closer (but not necessarily equal) to $a$. 

Now were going to add a new idea, the limit \textit{at infinity}. When
we write
\[
\lim_{x\to \infty} f(x) = L
\]
all we mean is that the value of $f(x)$ gets as close as we want to
$L$ when $x$ gets larger and larger. 

While the phrase ``at infinity'' might seem mystical or far-fetched,
it is actually very practical. Let's see if we can convince you of
this. Recently I started living with a puppy:

\begin{image}
\includegraphics{puppy.png}
\end{image}

I want to know how big this puppy will be when he grows up. It turns
out that growth of animals can be modeled. With some work I found that
this puppy's growth is modeled by:

\[
h(t) = \frac{45 e^{4.33 t}}{12+3 e^{4.33 t}}
\]

This gives the ``height'' of the puppy after $t$ years. For dogs, the
height of the dog is the ``shoulder height'' meaning the measurement
of how far the shoulder of the dog is from the floor when the dog is
standing.

If I want to know the puppy's adult height I would have several
options. Perhaps the most obvious thing to do is plug in an ``adult''
age and see what the formula gives. However, to do this, I would need
a calculator.

Can I somehow figure out dog's adult height using the formula
\emph{without} using a calculator? Here is where limits at infinity
come into play.


\begin{problem}
What does the notation
\[
t\to \infty
\]
mean?
\begin{multipleChoice}
\choice[correct]{$t$ becomes larger and larger.}
\choice{$t = \infty$.}
\choice{$t$ becomes every number at once.}
\end{multipleChoice}
\end{problem}

Consider our equation for the puppy's height:


\[
h(t) = \frac{45 e^{4.33 t}}{12+3 e^{4.33 t}}
\]

\begin{problem}
Which of the following are equivalent to 

\[
h(t) = \frac{45 e^{4.33 t}}{12+3 e^{4.33 t}}?
\]

\begin{multipleChoice}
\choice[correct]{$\frac{45}{12e^{-4.33t}+3}$}
\choice{$\frac{45 e^{4.33 t}}{12}\cdot \frac{45e^{4.33t}}{3 e^{4.33 t}}$}
\choice{$\frac{45 e^{4.33 t}}{12}+\frac{45e^{4.33t}}{3 e^{4.33 t}}$}
\end{multipleChoice}
\end{problem}

\begin{problem}
As $t\to \infty$ what happens to $e^{-4.33t}$?
\begin{multipleChoice}
\choice[correct]{$e^{-4.33t}$ approaches zero.}
\choice{$e^{-4.33t}$ approaches one.}
\choice{$e^{-4.33t}$ approaches infinity.}
\choice{$e^{-4.33t}$ approaches negative infinity.}
\choice{$e^{-4.33t}$ approaches $4.33$.}
\choice{$e^{-4.33t}$ approaches $-4.33$.}
\end{multipleChoice}
\end{problem}


\begin{problem}
What is 
\[
\lim_{t\to\infty} h(t)?
\]
\begin{multipleChoice}
\choice[correct]{$15$}
\choice{$3.75$}
\choice{$3$}
\choice{Infinity.}
\choice{Zero.}
\choice{Negative infinity.}
\end{multipleChoice}
\end{problem}

Part of learning calculus is about understanding and knowing
functions. Limits at infinity help us ``remove'' irrelevant data and
easily gain information from complex formulas.

\begin{problem}
What other questions do you have about this lecture?
\begin{freeResponse}
\end{freeResponse}
\end{problem}

\end{document}
