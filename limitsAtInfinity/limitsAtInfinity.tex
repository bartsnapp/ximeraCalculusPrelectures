\documentclass{ximera}

\outcome{What does it mean to take the limit as $x$ approaches $\infty$?}
\outcome{Determinate and indeterminate forms.}
\outcome{Why is infinity not a number?}
\outcome{Calculate the limit as $x$ approaches $\pm\infty$ of common functions algebraically.}
\outcome{Decide whether a form is determinate or indeterminate.}
\outcome{Find the limit as $x$ approaches $\pm\infty$ from a graph.}

\title{Limits at infinity}
\begin{document}

\begin{abstract}
  This activity will introduce students to the idea of a limit at infinity.
\end{abstract}
\maketitle

So far, when we study limits, we typically have something like this:
\[
\lim_{x\to a} f(x) = L
\]
This means that we can get the value of $f(x)$ as close as we want to
$L$, by moving $x$ closer and closer (but not necessarily equal) to $a$. 

Now were going to add a new idea, the limit \textit{at
  infinity}. While the phrase ``at infinity'' might seem mystical or
far-fetched, it is actually very practical. When we write
\[
\lim_{x\to \infty} f(x) = L
\]
all we mean is that the value of $f(x)$ gets as close as we want to
$L$ when $x$ gets larger and larger. 

\begin{question}
What does the 
\[
x\to \infty
\]
mean?
\begin{solution}
\begin{multiple-choice}
\choice[correct]{$x$ becomes larger and larger.}
\choice{$x = \infty$.}
\choice{$x$ becomes every number at once.}
\end{multiple-choice}
\end{solution}
\end{question}


Believe it or not, limits at infinity are usually a way to
\textit{simplify} something that is complex.

maybe rolling cylinder
Dog growing
relativistic speeds


Let me tell you about \textit{Project Excelsior}. \textit{Project
  Excelsior}

%\href{http://en.wikipedia.org/wiki/Project_Excelsior} \

was initiated in 1958 to study high altitude ejection systems. The plan
was this: Joe Kittinger would ride a balloon up to an extreme altitude
(31333 meters) and then jump from the balloon's gondola. Here is a
video: 

%\href{https://www.youtube.com/watch?v=fM39Z3grkpg}


The first 
Kittinger held the world record for high altitude jumps until 2012,
when Felix Baumgartner jumped from a balloon at an altitude of 38969
meters as part of the \textit{Red Bull Stratos} project.

After 


TERMINAL VELOCITY!


Penny from skyscraper. 

%http://www.scientificamerican.com/article/could-a-penny-dropped-off/

bass guitar string

%https://www.youtube.com/watch?v=hvl3ZCIzEK0

\begin{question}
  What is the correct answer to this question?

  \begin{solution}
    \begin{multiple-choice}
      \choice[correct]{Correct answer}
      \choice{First Distractor}
      \choice{Second Distractor}
      \choice{Third Distractor}
    \end{multiple-choice}  
  \end{solution}
\end{question}

What other questions do you have about this lecture?
\begin{free-response}
\end{free-response}

\end{document}
