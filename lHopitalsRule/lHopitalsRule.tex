\documentclass{ximera}

\outcome{What is an indeterminate form?}
\outcome{When do you have an indeterminate form?}
\outcome{How can we use derivatives to help us find limits?}
\outcome{What is L'Hopital's Rule and when can you use it?}
\outcome{Convert indeterminate forms to $0/0$ or $\infty/\infty$}
\outcome{Use L'Hopital's Rule to find limits.}
\outcome{Recall how to find limits for forms that are not indeterminate.}
\outcome{Determine if a form is indeterminate.}

\title{L'Hopitals rule}

\begin{document}

\begin{abstract}
   L'Hopital's rule lets us evaluate some limits using differential calculus  
\end{abstract}

\maketitle

\begin{theorem}
	Let $f$ and $g$ be functions which are differentiable on a neighborhood of $a$, with $g'(a) \neq 0$ in a neighborhood of $a$. 

	If the limit $\displaystyle\lim_{x \to a} \frac{f(x)}{g(x)}$ is of indeterminate form $\frac{0}{0}$ (in other words, in this case, $f(a) = g(a)=0$), then we have
	
	$\displaystyle\lim_{x \to a} \frac{f(x)}{g(x)} = \displaystyle\lim_{x \to a} \frac{f'(x)}{g'(x)}$
\end{theorem}

\begin{question}
  Does L'Hopital's Rule apply to $\lim_{x \to 2}\frac{x^3-2x-4}{x-2}$?
  
  \begin{solution}
 \begin{multipleChoice}
      \choice[correct]{Yes}
      \choice{No}
    \end{multipleChoice}
 \end{solution}
 
 Right! The limit is of indeterminate form $\frac{0}{0}$, both the numerator and denominator of the expression $\frac{x^3-2x-4}{x-2}$ are differentiable everywhere, and the derivative of the denominator is $1$, which is never equal to $0$.
 
 \end{question}
 
 \begin{question}
 	\begin{hint}
		Differentiate the numerator and denominator, then plug in $x=2$
	\end{hint}
  	 $\lim_{x \to 2}\frac{x^3-2x-4}{x-2}=$\answer{10}
  \end{question}
  
\begin{question}
  Does L'Hopital's Rule apply to $\lim_{x \to 0}\frac{x+3}{x+4}$?
  
  \begin{solution}
 \begin{multipleChoice}
      \choice{Yes}
      \choice[correct]{No}
    \end{multipleChoice}
 \end{solution}
 
 Right! The limit is not of indeterminate form $\frac{0}{0}$, so L'Hopital's rule does not apply.
 
 \end{question}
 
 \begin{question}
 	\begin{hint}
		A rational function is continuous on its domain.
	\end{hint}
  	 $\lim_{x \to 0}\frac{x+3}{x+4}=$\answer{3/4}
  \end{question}  
  

Write down at least \textbf{five} questions for this lecture. After
you have your questions, label them as ``Level 1,'' ``Level 2,'' or ``Level 3'' where:
\begin{description}
\item[Level 1] Means you know the answer, or know exactly how to do this problem.
\item[Level 2] Means you think you know how to do the problem, or will soon learn how to do the problem.
\item[Level 3] Means you have no idea how to do the problem. 
\end{description}
\begin{question}
  \begin{freeResponse}
  \end{freeResponse}
\end{question}

\end{document}
