\documentclass{ximera}

\outcome{How do we ``undo'' derivatives?}
\outcome{Why is undoing the derivative harder than taking one?}
\outcome{What is an indefinite integral?}
\outcome{What is an initial value problem?}
\outcome{Can we find position from velocity or acceleration?}
\outcome{Compute basic antiderivatives.}
\outcome{Solve basic initial value problems.}
\outcome{Use antiderivatives to solve simple word problems.}

\title{antiderivatives}

\begin{document}

\begin{abstract}
  $F$ is an antiderivative of $f$ if $F'=f$
\end{abstract}

\maketitle

\begin{definition}
	A function $F$ is called an \textbf{antiderivative} of a function $f$ if $F' = f$.  We write $\int f(x) dx$ for the set of all antiderivatives of $f$.
\end{definition}

\begin{question}
	$F(x) = x^2$ is one antiderivative of $f(x) = 2x$, since $F'(x) = f(x)$.  Does $f$ have any other antiderivatives?
\begin{solution}
	 \begin{multipleChoice}
    	  \choice[correct]{Yes}
     	 \choice{No}
    	\end{multipleChoice}  
  \end{solution}
  
  Right!  For example, $F(x) = x^2+1$ also has derivative $F'(x) = 2x$.
\end{question}

\begin{theorem}
	If $f$ is differentiable on an interval $[a,b]$, then all of the antiderivatives of $f$ differ by a constant on that interval.  For this reason we will always write $\int f(x) dx  = F(x)+C$, where $F$ is a particular antiderivative of $f$ and $C$ is an arbitrary constant.
\end{theorem}

\begin{question}
 	$\int 4x dx = $ 
	\begin{expression-answer}
 function validator(F) {
   if (F.variables().indexOf('C') == -1) {
     feedback( 'You should include a $+C$ in your answer.' );
     return false;
   }
   
   var correctf = MathFunction.parse('4x')
   var dF = F.derivative('x');
   
   return (dF.equals(correctf));
 }
\end{expression-answer}	
	
\end{question}

\begin{question}
Write down at least \textbf{five} questions for this lecture. After
you have your questions, label them as ``Level 1,'' ``Level 2,'' or ``Level 3'' where:
\begin{description}
\item[Level 1] Means you know the answer, or know exactly how to do this problem.
\item[Level 2] Means you think you know how to do the problem, or will soon learn how to do the problem.
\item[Level 3] Means you have no idea how to do the problem. 
\end{description}
\begin{freeResponse}
\end{freeResponse}
\end{question}

\end{document}
