\documentclass{ximera}

\outcome{Put together everything learned in this class so far to determine how the graph of a function looks without using a graphing calculator.}

\title{graphing functions}

\begin{document}

\begin{abstract}
  We can now use calculus to graph functions by hand
\end{abstract}

\maketitle




This prefecture is one long extended exercise.  We will learn everything we can about the function $f(x) = x^3-12x$, and use this information to graph $f$.
\textbf{Do not} use a graphing calculator to help you during this activity.  It will rob you of the whole experience.



\textbf{Domain}

\begin{question}
What is the domain of $f$?
 
    \begin{multipleChoice}
      \choice[correct]{$\mathbb{R}$}
      \choice{$(0,\infty)$}
      \choice{$(-12,12)$}
      \choice{$(-\sqrt{12},\sqrt{12})$}
      \choice{DNE}
    \end{multipleChoice}
    
\end{question}    


 \textbf{Find First and Second Derivative}
 
 \begin{question}
 	$f'(x)=$\answer{3x^2-12}
 \end{question}
 
 \begin{question}
 	$f''(x)=$\answer{6x}
 \end{question}
 
 \textbf{Find the Critical Points and Possible Inflection Points}
 
 \begin{question}
	Now find all of the points where $f'(x)=0$ or does not exist (critical points), and where $f''(x)=0$ or does not exist (possible inflection points).
	
	Which of the following sets is the set of all critical points and possible inflection points of $f$ on the interval $[0,\infty)$?
	
	 \begin{multipleChoice}
     \choice[correct]{$\{-2,0,2\}$}
      \choice{$\{-2,2\}$}
      \choice{$\{0\}$ }
      \choice{$\{-12,12\}$}
      \choice{There are no critical points or possible inflection points}
    \end{multipleChoice}

 \end{question}
 
 \textbf{Identify Increasing/Decreasing Behavior and Convaity}
 
 \begin{question}

On the interval $(-\infty,-2)$ $f$ is
 
 \begin{multipleChoice}
     \choice{Decreasing and concave down}
      \choice{Decreasing and concave up}
      \choice[correct]{Increasing and concave down}
      \choice{Increasing and concave up}
    \end{multipleChoice} 
    
    On the interval $(-2,0)$ $f$ is
 
 \begin{multipleChoice}
     \choice[correct]{Decreasing and concave down}
      \choice{Decreasing and concave up}
      \choice{Increasing and concave down}
      \choice{Increasing and concave up}
    \end{multipleChoice}

On the interval $(0,2)$ $f$ is
 
 \begin{multipleChoice}
     \choice{Decreasing and concave down}
      \choice[correct]{Decreasing and concave up}
      \choice{Increasing and concave down}
      \choice{Increasing and concave up}
    \end{multipleChoice}
    \end{question}
    
    On the interval $(2,\infty)$ $f$ is
 
 \begin{multipleChoice}
     \choice{Decreasing and concave down}
      \choice{Decreasing and concave up}
      \choice{Increasing and concave down}
      \choice[correct]{Increasing and concave up}
    \end{multipleChoice}
    \end{question}
    
 \textbf{Identify extreme values and inflection points}
	
	\begin{question}
		$f$ has a local min at $x=$\answer{2}
	\end{question} 
	
	\begin{question}
		$f$ has a local max at $x=$\answer{-2}
	\end{question} 
	
	\begin{question}
		$f$ has an inflection point at $x=$\answer{0}
	\end{question} 
 
 \textbf{Locate vertical/horizontal asymptotes and determine end behavior}
 
 \begin{question}
 Which of the following are true?
 
 \begin{multipleChoice}
     \choice[correct]{$\lim_{x \to -\infty} f(x) = -\infty$ and $\lim_{x \to \infty} f(x) = \infty$}
      \choice{$\lim_{x \to -\infty} f(x) = \infty$ and $\lim_{x \to \infty} f(x) = \infty$}
      \choice{$\lim_{x \to -\infty} f(x) = \infty$ and $\lim_{x \to \infty} f(x) = -\infty$}
      \choice{$\lim_{x \to -\infty} f(x) = 0$ and $\lim_{x \to \infty} f(x) = 0$}
    \end{multipleChoice}
    \end{question}
    
    
  \textbf{Find the Intercepts}
  
 In order, from smallest value of $x$ to largest value of $x$, 
 
 \begin{question}
 	The first zero of $f$ is $x=-sqrt(12)$
 \end{question}
 
  \begin{question}
 	The second zero of $f$ is $x=0$
 \end{question}
 
  \begin{question}
 	The last zero of $f$ is $x=sqrt(12)$
 \end{question}
 
  
  \textbf{Choose an Appropriate Graphing Window and Make a Graph}
  
  Try to graph the function now, using all of the information you have gathered above.  
  
  
  \begin{question}
  
  Do you solemnly swear that you tried to graph it on your own?  We will reveal the graph after you say "yes", so that you can compare your work with the correct answer.
 
 \begin{solution} 
  	\begin{multipleChoice}
     \choice[correct]{yes}
      \choice{no}
    \end{multipleChoice}
\end{solution}

\begin{image}
  \begin{tikzpicture}
    \colorlet{textColor}{black}
    \colorlet{penColor}{blue!50!black}
	\begin{axis}[
            domain=-5:5,
            ymax=20,
            ymin=-20,
            width=6in,
            %samples=100,
            axis lines =middle, xlabel=$x$, ylabel=$y$,
            every axis y label/.style={at=(current axis.above origin),anchor=south},
            every axis x label/.style={at=(current axis.right of origin),anchor=west}
          ]
      
          \addplot [thick, penColor, smooth,domain=(-4:4)] {x/(x^2-1)};
       
        \end{axis}
\end{tikzpicture}
\end{image}
  \end{question}
  
       


Note that this graph is symmetric about the origin, since the function is odd (check this yourself!)

  
\begin{question}
  Write down at least \textbf{five} questions for this lecture. After
you have your questions, label them as ``Level 1,'' ``Level 2,'' or ``Level 3'' where:
\begin{description}
\item[Level 1] Means you know the answer, or know exactly how to do this problem.
\item[Level 2] Means you think you know how to do the problem, or will soon learn how to do the problem.
\item[Level 3] Means you have no idea how to do the problem. 
\end{description}
  \begin{freeResponse}
  \end{freeResponse}
  \end{question}
  



\end{document}
