\documentclass{ximera}

%\colorlet{penColor}{blue!50!black} % Color of a curve in a plot
%\colorlet{gridColor}{gray!50} % Color of grid in a plot
%\colorlet{background}{white} % Color of the page
\newcommand{\ddx}{\frac{d}{dx}}
\newcommand{\dydx}{\frac{dy}{dx}}
\newcommand{\dd}[2][]{\frac{d #1}{d #2}}


\outcome{How does changing the variable change how we take the derivative?}
\outcome{Understanding the derivatives of expressions that are not functions or not solved for $y$.}
\outcome{Implicitly differentiate expressions.}
\outcome{Solve equations for $dy/dx$.}
\outcome{Find the equation of the tangent line for curves that are not functions.}
\outcome{Find derivatives of functions with rational exponents.}

\title{Implicit differentiation}

\begin{document}

\begin{abstract}
 Implicit differentiation allows us to take the derivative of an
 equation.
\end{abstract}
\maketitle

The functions we've been dealing with so far have been
\textit{explicit functions}, meaning that the
dependent variable is written in terms of the independent
variable. For example:

\[
y=3x^2-2x+1,\qquad y=e^{3x}, \qquad y = \frac{x-2}{x^2-3x+2}.
\]

However, there are another type of functions, called \textit{implicit
  functions}. In this case, the dependent variable is not stated
explicitly in terms of the independent variable. For example:

\[
x^2+y^2 = 4,\qquad x^3+y^3 = 9xy, \qquad x^4+3x^2 = x^{2/3}+y^{2/3} = 1.
\]

Your inclination might be simply to solve each of these for $y$ and go
merrily on your way. However this can be difficult and it may require
two \textit{branches}, for example to explicitly plot $x^2+y^2 = 4$,
one needs both $y= \sqrt{4-x^2}$ and $y=-\sqrt{4-x^2}$. Moreover, it
may not even be possible to solve for $y$. To deal with such
situations, we use \textit{implicit differentiation}. Let's see an
illustrative example: Consider the curve defined by
\[
x^3+y^3 = 9xy.
\]


\begin{tikzpicture}
  \begin{axis}[
      xmin=-6,xmax=6,ymin=-6,ymax=6,
      width=5in,
      axis lines=center,
      xlabel=$x$, ylabel=$y$,
      every axis y label/.style={at=(current axis.above origin),anchor=south},
      every axis x label/.style={at=(current axis.right of origin),anchor=west},
    ]        
    \addplot [very thick, blue!50!black, smooth, samples=100, domain=(-.99:0)] ({9*x/(1+x^3)},{9*x^2/(1+x^3)});
    \addplot [very thick, blue!50!black, smooth, samples=100, domain=(-.99:0)] ({9*x^2/(1+x^3)},{9*x/(1+x^3)});
    \addplot [very thick, blue!50!black, smooth, samples=100, domain=(0:1)] ({9*x/(1+x^3)},{9*x^2/(1+x^3)});
    \addplot [very thick, blue!50!black, smooth, samples=100, domain=(0:1)] ({9*x^2/(1+x^3)},{9*x/(1+x^3)});
  \end{axis}
\end{tikzpicture}

Let's try to compute $\dydx$:


\begin{question}
We start by taking the derivative of the left-hand side of 
\[
x^3+y^3 = 9xy.
\]
However, when we do this we treat $y$ as a function of $x$. This means:
\[
\ddx y = \dydx.
\]
Compute 
\[
\ddx \left(x^3+y^3\right).
\]
\begin{hint}
Use the chain rule.
\end{hint}
\begin{prompt}
$\ddx \left(x^3+y^3\right) = $\answer{3*x^2 + 3*y^2 *dy/dx}
\end{prompt}
\end{question}

\begin{question}
Now we take the derivative of the right-hand side of 
\[
x^3+y^3 = 9xy.
\]
However, again when we do this we treat $y$ as a function of $x$. This
means:
\[
\ddx y = \dydx.
\]
Compute 
\[
\ddx 9xy
\]
\begin{hint}
Use the product rule.
\end{hint}
\begin{prompt}
$\ddx 9xy= $\answer{9*x*dy/dx + 9y}
\end{prompt}
\end{question}


\begin{question}
Now put our two sides together and solve for $\dydx$:
\begin{hint}
We now have 
\[
3x^2 + 3y^2\dydx= 9x\dydx + 9y,
\]
solve for $\dydx$.
\end{hint}
\begin{prompt}
$\dydx=$\answer{(3x^2-9y)/(9x-3y^2)}.
\end{prompt}
\end{question}


\begin{question}
What is the slope of the tangent line when $x=4$ and $y=2$?
\begin{prompt}
The slope is \answer{5/4}.
\end{prompt}
\end{question}


\begin{question}
Write down at least \textbf{five} questions for this lecture. After
you have your questions, label them as ``Level 1,'' ``Level 2,'' or ``Level 3'' where:
\begin{description}
\item[Level 1] Means you know the answer, or know exactly how to do this problem.
\item[Level 2] Means you think you know how to do the problem, or will soon learn how to do the problem.
\item[Level 3] Means you have no idea how to do the problem. 
\end{description}
  \begin{freeResponse}
  \end{freeResponse}
\end{question}

\end{document}
