\documentclass{ximera}

\colorlet{penColor}{blue!50!black} % Color of a curve in a plot
\colorlet{gridColor}{gray!50} % Color of grid in a plot
\colorlet{background}{white} % Color of the page


\outcome{What is linear approximation and when can it be used?}
\outcome{Understand how good an approximation is.}
\outcome{What is a differential?}
\outcome{Find the linear approximation to a function at a point and use it to approximate the function value.}
\outcome{Find the error of a linear approximation.}
\outcome{Calculate $\delta y$ and $dy$.}

\title{Linear approximation}

\begin{document}

\begin{abstract}
  Now we see the derivatives connection to linear approximations. 
\end{abstract}
\maketitle

Given a function, a \textit{linear approximation} is a fancy phrase
for something you already know.

\begin{definition}\index{linear approximation}
If $f(x)$ is a differentiable function at $x=a$, then a \textbf{linear
  approximation} for $f(x)$ at $x=a$ is given by
\[
\l(x) = f'(a)(x-a) +f(a).
\]
\end{definition}


Let's use a linear approximation of $f(x) =\sqrt[3]{x}$ at $x=64$ to
approximate $\sqrt[3]{50}$.

\begin{question}
To start, $\ddx \sqrt[3]{x} = $\answer{3*x^(2/3)}
\end{question}

Now we need to find a whole number whose cube root is ``easy'' that is
near $50$.


\begin{question}
What is the closest whole number to $50$ that is a perfect cube?
\answer{64}
\end{question}

\begin{question}
Now we write the formula for the tangent line to the curve at $x=64$.

The formula for the tangent line is $\l(x) = $\answer{x/48 + 8/3}.
\end{question}

\begin{question}
  $\l(50) =$\answer{3.71} (round to two decimal places).
\end{question}

Now check your answer:

\begin{question}
$(\text{your previous answer})^3$ = \answer{51.06} (round to two
  decimal places).
\end{question}




With modern calculators and computing software it may not appear
necessary to use linear approximations. But in fact they are quite
useful. In cases requiring an explicit numerical approximation, they
allow us to get a quick rough estimate which can be used as a
``reality check'' on a more complex calculation. In some complex
calculations involving functions, the linear approximation makes an
otherwise intractable calculation possible, without serious loss of
accuracy.


\begin{question}
Write down at least \textbf{five} questions for this lecture. After
you have your questions, label them as ``Level 1,'' ``Level 2,'' or ``Level 3'' where:
\begin{description}
\item[Level 1] Means you know the answer, or know exactly how to do this problem.
\item[Level 2] Means you think you know how to do the problem, or will soon learn how to do the problem.
\item[Level 3] Means you have no idea how to do the problem. 
\end{description}
  \begin{freeResponse}
  \end{freeResponse}
\end{question}

\end{document}
