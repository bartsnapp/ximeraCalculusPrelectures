\documentclass{ximera}

%\colorlet{penColor}{blue!50!black} % Color of a curve in a plot
%\colorlet{gridColor}{gray!50} % Color of grid in a plot
%\colorlet{background}{white} % Color of the page
\newcommand{\ddx}{\frac{d}{dx}}
\newcommand{\dydx}{\frac{dy}{dx}}
\newcommand{\dd}[2][]{\frac{d #1}{d #2}}


\outcome{Using the definition of the derivative to develop ``shortcut'' rules to find the derivatives of: constants, constant multiples, powers of $x$, sums of functions, and natural exponential functions.}
\outcome{What is a higher order derivative?}
\outcome{Use ``shortcut'' rules to find and use derivatives.}
\outcome{Calculate higher order derivatives.}

\title{Rules of differentiation}

\begin{document}

\begin{abstract}
  In this activity, we begin to unlock the ``secret'' of the derivative. 
\end{abstract}
\maketitle


The simplest function is a constant function.  Recall that derivatives
measure the rate of change of a function at a given point. Hence, the
derivative of a constant function is zero. For example:
\begin{itemize}
\item The constant function plots a horizontal line---so the slope of
  the tangent line is $0$.
\item If $p(t)$ represents the position of an object with respect to
  time and $p(t)$ is constant, then the object is not moving, so its
  velocity is zero. Hence $\dd{t} p(t) = 0$.
\item If $v(t)$ represents the velocity of an object with respect to
  time and $v(t)$ is constant, then the object's acceleration is
  zero. Hence $\dd{t} v(t) = 0$.
\end{itemize}


\begin{question}
  What is the derivative with respect to $x$ of $f(x) = e$? 
\begin{hint}
Remember, $e=2.718281828459045\dots$.
\end{hint}
    \begin{multipleChoice}
      \choice[correct]{$0$ because the derivative of a constant is $0$.}
      \choice{$e$ because the derivative of $e$ is $e$.}
      \choice{$1$ because $e^0=1$.}
      \choice{Nobody knows the answer to this question.}
    \end{multipleChoice}  
\end{question}

Now let's examine derivatives of powers of a single variable.  Here we
have a nice rule.

\begin{theorem}[The Power Rule]
For any real number $n$, 
\[
\ddx x^n = n x^{n-1}.
\]
\end{theorem}

\begin{question}
  What is the derivative of $f(x) = x^e$ with respect to $x$?
\begin{prompt}
$\ddx f(x) = $\answer{e*x^(e-1)}
\end{prompt}
\end{question}


However, both of these rules are best in conjection with the \textit{sum rule}.

\begin{theorem}[The Sum Rule]
If $f(x)$ and $g(x)$ are differentiable and $c$ is a constant, then 
\begin{enumerate}
\item $\ddx \big( f(x) + g(x)\big) = f'(x) + g'(x)$,
\item $\ddx \big( f(x) - g(x)\big) = f'(x) - g'(x)$,
\item $\ddx \big(c\cdot f(x)\big) = c\cdot f'(x)$.
\end{enumerate}
\end{theorem}

With these rules combined, we can compute the derivatives of polynomials.


\begin{question}
  What is the derivative of $f(x) = 7x^3-3x^2+1$ with respect to $x$?
\begin{prompt}
$\ddx f(x) = $\answer{21x^2-6x}
\end{prompt}
\end{question}

Finally we have the rule for $e^x$.

\begin{theorem}[The Derivative of $\textit{e}^\textit{x}$]
\[
\ddx e^x = e^x.
\]
\end{theorem}

\begin{question}
  What is the derivative of $f(x) = e^x$ with respect to $x$?
\begin{prompt}
$\ddx f(x) = $\answer{e^x}
\end{prompt}
\end{question}

\begin{question}
Write down at least \textbf{five} questions for this lecture. After
you have your questions, label them as ``Level 1,'' ``Level 2,'' or ``Level 3'' where:
\begin{description}
\item[Level 1] Means you know the answer, or know exactly how to do this problem.
\item[Level 2] Means you think you know how to do the problem, or will soon learn how to do the problem.
\item[Level 3] Means you have no idea how to do the problem. 
\end{description}
  \begin{freeResponse}
  \end{freeResponse}
\end{question}

\end{document}
